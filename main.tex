\documentclass[
    % english, % Klasei padavus parametrą 'english', darbas bus anglų kalba.
    % signatureplaces % prideda parašų vietas tituliniame puslapyje
]{VUMIFPSbakalaurinis}
\usepackage{float}
\usepackage{wrapfig2}
\usepackage{hyperref}
\usepackage{algorithmicx}
\usepackage{algorithm}
\usepackage{algpseudocode}
\usepackage{amsfonts}
\usepackage{amsmath}
\usepackage{bm}
\usepackage{caption}
\usepackage{color}
\usepackage{graphicx}
\usepackage{listings}
\usepackage{subcaption}
\usepackage{biblatex}

\university{Vilniaus universitetas}
\faculty{Matematikos ir informatikos fakultetas}
\department{Informatikos bakalauro studijų programa}
\title{Dviejų pasirinktų architektūrų palyginimas}
\papertype{Motorola 68HC11 ir Intel i960}

\author{Ainis Augustas Laurinavičius}
\supervisor{prof. dr. Saulius Gražulis}
\date{Vilnius – \the\year}

\bibliography{bibliografija}

\begin{document}
\maketitle

\newpage
\section*{2. Punktas}
\subsection*{Motorola 68HC11}
Šio procesoriaus bazė yra padaryta iš CMOS MOSFET tranzistorių, integrinio grandyno (IC), kuris yra labai didelio integracijos masto (VLSI). Šis procesorius yra monokristalinis šiuolaikinis procesorius. Procesorius yra apie 20mm ilgio ir 20mm pločio, svoris nenurodytas. Vykdymo metu procesorius vartoja 5V, 2MHz dažniu, o stipris svyruoja tarp 5 - 15mA. Padidinus veikimo dažnį iki 3 - 4MHz, stipris svyruoja nuo 8 iki 20mA. Procesorius STOP režime (low-power) naudoja mažiau arba apie 10$\mu$A, o miego režime (WAIT) stipris tiksliai nenurodytas, procesoriaus galia vykdymo metu yra tarp 50 - 100mW.

\subsection*{Intel i960}
Šio procesoriaus bazė yra sudaryta iš CMOS tranzistorių, integrinio grandyno, kuris yra labai didelio integracijos masto (VLSI). Šis procesorius yra monokristalinis šiuolaikinis procesorius. Procesorius yra apie 35mm ilgio ir 35mm pločio, sveriantis apie 50g.Vykdymo metu procesorius vartoja 3.3V, 25MHz dažniu, o stipris apie 150mA. Padidinus veikimo dažnį iki 33MHz, stipris padidėja iki 200mA. Procesorius STOP režime (low-power) naudoja tik 10mA - 20mA, o miego režime (WAIT) 37mA - 75mA, procesoriaus galia vykdymo metu yra apie 0.033W.

\end{document}