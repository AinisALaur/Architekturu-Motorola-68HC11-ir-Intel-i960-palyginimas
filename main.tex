\documentclass[
    % english, % Klasei padavus parametrą 'english', darbas bus anglų kalba.
    % signatureplaces % prideda parašų vietas tituliniame puslapyje
]{VUMIFPSbakalaurinis}
\usepackage{float}
\usepackage{wrapfig2}
\usepackage{hyperref}
\usepackage{algorithmicx}
\usepackage{algorithm}
\usepackage{algpseudocode}
\usepackage{amsfonts}
\usepackage{amsmath}
\usepackage{bm}
\usepackage{caption}
\usepackage{color}
\usepackage{graphicx}
\usepackage{listings}
\usepackage{subcaption}
\usepackage{biblatex}

\university{Vilniaus universitetas}
\faculty{Matematikos ir informatikos fakultetas}
\department{Informatikos bakalauro studijų programa}
\title{Dviejų pasirinktų architektūrų palyginimas}
\papertype{Motorola 68HC11 ir Intel i960}

\author{Ainis Augustas Laurinavičius}
\supervisor{prof. dr. Saulius Gražulis}
\date{Vilnius – \the\year}

\bibliography{bibliografija}

\begin{document}
\maketitle

\newpage
\section*{Elementinė procesoriaus/kompiuterio bazė, fizinės įrangos savybės}
\subsection*{Motorola 68HC11}
Šios architektūros bazė yra padaryta iš CMOS MOSFET tranzistorių, integrinio grandyno (IC), kuris yra labai didelio integracijos masto (VLSI). Šis procesorius yra monokristalinis šiuolaikinis procesorius. Procesorius yra apie 20mm ilgio ir 20mm pločio, svoris nenurodytas. Vykdymo metu procesorius vartoja 5V, 2MHz dažniu, o stipris svyruoja tarp 5 - 15mA. Padidinus veikimo dažnį iki 3 - 4MHz, stipris svyruoja nuo 8 iki 20mA. Procesorius STOP režime (low-power) naudoja mažiau - apie 10$\mu$A, o miego režime (WAIT) stipris tiksliai nenurodytas, procesoriaus galia vykdymo metu yra tarp 50 - 100mW.

\subsection*{Intel i960}
Šios architektūros bazė yra sudaryta iš CMOS tranzistorių, integrinio grandyno, kuris yra labai didelio integracijos masto (VLSI). Šis procesorius yra monokristalinis šiuolaikinis procesorius. Procesorius yra apie 35mm ilgio ir 35mm pločio, sveriantis apie 50g.Vykdymo metu procesorius vartoja 3.3V, 25MHz dažniu, o stipris apie 150mA. Padidinus veikimo dažnį iki 33MHz, stipris padidėja iki 200mA. Procesorius STOP režime (angl. "low-power") naudoja tik 10mA - 20mA, o miego režime (angl. "WAIT") 37mA - 75mA, procesoriaus galia vykdymo metu yra apie 0.033W.

\section*{Kokio tipo architektūrą turėjo abu kompiuteriai}
\subsection*{Motorola 68HC11}
Ši architektūra turėjo 8-bitų akumuliatorių, steko rodyklę (angl. "stack pointer") (bet jo nenaudoja operacijoms), programos skaitliuką, registrus, kurie palaiko 72 instrukcijas ir 19 išplėstinių operacijos kodų (angl. "opcodes"), visos operacijos vykdomos tarp registrų ar akumuliatoriuje, o ne atmintyje, tad šio procesoriaus architektūra yra akumuliatorinė/registrinė.

\subsection*{Intel i960}
Ši architektūra yra sudaryta registrų, turi steko rodyklę (angl. "stack pointer") (bet jo nenaudoja operacijoms), neturi akumuliatoriaus, naudoja įkelti, saugoti instrukcijas (angl. "load/store instructions"), visos operacijos vykdomos tarp registrų, o ne atmintyje, tad šio procesoriaus architektūra yra registrinė, kuri naudoja įkelti, saugoti instrukcijas.

\section*{Ar tai buvo beadresinės (stekinės), vieno adreso, dviejų adresų, trijų adresų ar keturios adresų mašinos?}
\subsection*{Motorola 68HC11}
Ši architektūra yra 1 - 2 adresų mašina, kurios adresų kiekis priklauso nuo operacijos, o daugiau kaip 2 adresų nenaudoja. Architektūra turi steką, bet jis nėra naudojamas operacijoms ("The M68HC11 CPU automatically supports a program stack… used for subroutines, interrupts, and temporary storage." citata iš "Motorola 68HC11 reference manual").

\subsection*{Intel i960}
Ši architektūra naudoja 2 - 3 adresus, priklausomai nuo operacijos: loginės operacijos naudoja 3 adresus, o operacijos kaip įkelti, saugoti (angl. "load, store") naudoja tik 2 adresus, vieno adreso operacijų architektūra neturi. Architektūra turi steką, bet jis nėra naudojamas operacijoms, citata iš "I960 INTEL Datasheet": "it does not consume/produce operands on a stack."

\section*{Kokie buvo registrai abiejose architektūrose? Kokie, tai registrai? Kiek registrų? Kokie registrų duomenų pločiai? Kokia buvo specifinė registrų paskirtis?}
\subsection*{Motorola 68HC11}
Architektūroje yra 8 registrai (A, B, D, X, Y, SP, PC, CCR) ir visi registrai yra globalūs. Registrai A, B ir CCR yra 8 bitų pločio, o visi kiti yra 16 bitų pločio. Registrai SP ir CCR yra specialios paskirties - SP yra steko rodyklės (angl. "stack pointer") registras, CCR yra sąlygos kodo registras (angl. "Condition Code Register"), o PC registras yra programos skaitliukas (angl. "Program Counter"). Registrai X ir Y yra skirti indeksavimui, bet galima naudoti, kaip bendros paskirties registrus, likę registrai (A, B, D) - yra bendros paskirties registrai.

\subsection*{Intel i960}
Architektūra sudaryta iš 32 registrų (16 globalių ir 16 lokalių bendros paskirties) ir visi registrai yra 32 bitų pločio. Specialios paskirties registrai yra: PC / IP (instrukcijų rodyklė (angl. "Instruction Pointer")), AC (aritmetinių sąlygų kodo registras (angl. "Arithmetic Condition Codes")), MMRs (valdo pertraukimus, skaitliukus, programos klaidas).

\section*{Ar požymių bitai buvo naudojami šiose architektūrose? Kokie požymiai buvo naudojami?}
\subsection*{Motorola 68HC11}
Taip ši architektūra naudoja požymių bitus - CCR registre yra 5 požymių, 2 pertraukimų ir 1 STOP bitas. Požymių bitai: C (perkėlimo (angl. "Carry \ Borrow")), V (perpildymas (angl. "Overflow")), Z (nulio), N (neigiamo), H (pusinio perkėlimo). Pertraukimo bitai I ir X ("Interrupt mask bits"). STOP bitas S - nutraukia programos veikimą.

\subsection*{Intel i960}
Taip ši architektūra taip pat naudoja požymių bitus. Požymių bitai: C (perkėlimo (angl. "Carry \ Borrow")), V (perpildymas (angl. "Overflow")), Z (nulio), N (neigiamo), T (sekimo įgalinimas (angl. "Trace-enable"), skirtas programos derinimui (ang."Debugging")). Taip pat yra klaidų kodų bitai: dalybą iš nulio, negalima prieiga, segmento klaida, lygiavimo klaida.

\end{document}