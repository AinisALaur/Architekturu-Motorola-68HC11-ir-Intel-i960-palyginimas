\documentclass[
    % english, % Klasei padavus parametrą 'english', darbas bus anglų kalba.
    % signatureplaces % prideda parašų vietas tituliniame puslapyje
]{VUMIFPSbakalaurinis}
\usepackage{float}
\usepackage{wrapfig2}
\usepackage{hyperref}
\usepackage{algorithmicx}
\usepackage{algorithm}
\usepackage{algpseudocode}
\usepackage{amsfonts}
\usepackage{amsmath}
\usepackage{bm}
\usepackage{caption}
\usepackage{color}
\usepackage{graphicx}
\usepackage{listings}
\usepackage{subcaption}
\usepackage{biblatex}

\university{Vilniaus universitetas}
\faculty{Matematikos ir informatikos fakultetas}
\department{Informatikos bakalauro studijų programa}
\title{Dviejų pasirinktų architektūrų palyginimas}
\papertype{Motorola 68HC11 ir Intel i960}

\author{Ainis Augustas Laurinavičius}
\supervisor{prof. dr. Saulius Gražulis}
\date{Vilnius – \the\year}

\bibliography{bibliografija}

\begin{document}
\maketitle

\newpage
\section*{2. Punktas}
\subsection*{Motorola 68HC11}
Tai yra 8-bitų mikrokontrolerių šeima, sukurta "Motorola Semiconductor".
Šio procesoriaus bazė yra padaryta iš CMOS MOSFET tranzistorių, turinti integrinį grandyną (IC), kuris yra labai didelio integracijos masto (VLSI). Šis procesorius yra monokristalinis šiuolaikinis procesorius, nes yra sudarytas iš vieno kristalo.

\end{document}