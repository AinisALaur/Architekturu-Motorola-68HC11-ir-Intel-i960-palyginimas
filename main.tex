\documentclass[
    % english, % Klasei padavus parametrą 'english', darbas bus anglų kalba.
    % signatureplaces % prideda parašų vietas tituliniame puslapyje
]{VUMIFPSbakalaurinis}
\usepackage{float}
\usepackage{wrapfig2}
\usepackage{hyperref}
\usepackage{algorithmicx}
\usepackage{algorithm}
\usepackage{algpseudocode}
\usepackage{amsfonts}
\usepackage{amsmath}
\usepackage{bm}
\usepackage{caption}
\usepackage{color}
\usepackage{graphicx}
\usepackage{listings}
\usepackage{subcaption}
\usepackage{biblatex}

\university{Vilniaus universitetas}
\faculty{Matematikos ir informatikos fakultetas}
\department{Informatikos bakalauro studijų programa}
\title{Dviejų pasirinktų architektūrų palyginimas}
\papertype{Motorola 68HC11 ir Intel i960}

\author{Ainis Augustas Laurinavičius}
\supervisor{prof. dr. Saulius Gražulis}
\date{Vilnius – \the\year}

\bibliography{bibliografija}

\begin{document}
\maketitle

\newpage
\section*{Elementinė procesoriaus/kompiuterio bazė, fizinės įrangos savybės}
\subsection*{Motorola 68HC11}
Šio procesoriaus bazė yra padaryta iš CMOS MOSFET tranzistorių, integrinio grandyno (IC), kuris yra labai didelio integracijos masto (VLSI). Šis procesorius yra monokristalinis šiuolaikinis procesorius. Procesorius yra apie 20mm ilgio ir 20mm pločio, svoris nenurodytas. Vykdymo metu procesorius vartoja 5V, 2MHz dažniu, o stipris svyruoja tarp 5 - 15mA. Padidinus veikimo dažnį iki 3 - 4MHz, stipris svyruoja nuo 8 iki 20mA. Procesorius STOP režime (low-power) naudoja mažiau - apie 10$\mu$A, o miego režime (WAIT) stipris tiksliai nenurodytas, procesoriaus galia vykdymo metu yra tarp 50 - 100mW.

\subsection*{Intel i960}
Šio procesoriaus bazė yra sudaryta iš CMOS tranzistorių, integrinio grandyno, kuris yra labai didelio integracijos masto (VLSI). Šis procesorius yra monokristalinis šiuolaikinis procesorius. Procesorius yra apie 35mm ilgio ir 35mm pločio, sveriantis apie 50g.Vykdymo metu procesorius vartoja 3.3V, 25MHz dažniu, o stipris apie 150mA. Padidinus veikimo dažnį iki 33MHz, stipris padidėja iki 200mA. Procesorius STOP režime (angl. "low-power") naudoja tik 10mA - 20mA, o miego režime (angl. "WAIT") 37mA - 75mA, procesoriaus galia vykdymo metu yra apie 0.033W.

\section*{Kokio tipo architektūrą turėjo abu kompiuteriai}
\subsection*{Motorola 68HC11}
Šis procesorius turėjo 8-bitų akumuliatorių, steko rodyklę (angl. "stack pointer") (bet jo nenaudoja operacijoms), programos skaitliuką, sąlyginius ir indeksų registrus, kurie palaiko 72 instrukcijas ir 19 išplėstinių operacijos kodų (angl. "opcodes"), visos operacijos vykdomos tarp registrų ar akumuliatoriuje, o ne atmintyje, tad šio procesoriaus architektūra yra akumuliatorinė/registrinė.

\subsection*{Intel i960}
Šis procesorius yra sudarytas iš 32 registrų (16 globalių ir 16 lokalių), turi steko rodyklę (angl. "stack pointer") (bet jo nenaudoja operacijoms), neturi akumuliatoriaus, naudoja įkelti, saugoti instrukcijas (angl. "load/store instructions"), visos operacijos vykdomos tarp registrų ar akumuliatoriuje, o ne atmintyje, tad šio procesoriaus architektūra yra registrinė, kuri naudoja įkelti, saugoti instrukcijas.
\end{document}